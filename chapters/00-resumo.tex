\chapter*{Resumo}

\noindent Kanashiro, L. \textbf{Teste em Larga Escala para Sistemas e
Aplica��es de Cidades Inteligentes}.
2017. 120 f. Exame de Qualifica��o (Mestrado) - Instituto de Matem�tica e Estat�stica,
Universidade de S�o Paulo, S�o Paulo, 2017.
\\~

\noindent
Cidades ao redor do mundo enfrentam diversos desafios para proporcionar uma boa
qualidade de vida a seus cidad�os. Sistemas e aplica��es de software v�m sendo
desenvolvidos com objetivo de melhorar os servi�os e otimizar o uso da
infraestrutura da cidade. Desenvolver ambientes de teste e experimenta��o para
esses softwares na escala de grandes cidades ainda � um desafio, devido ao
alto custo e problemas de infraestrutura. Neste trabalho ser� apresentado um
mecanismo de teste e experimenta��o de sistemas e aplica��es para Cidades
Inteligentes usando simula��o de larga escala. Esse mecanismo foi implementado
a partir da integra��o do InterSCSimulator e a plataforma InterSCity.\\~

\noindent \textbf{Palavras-chave}: Cidade Inteligentes, Teste de Aplica��es, Experimenta��o.

